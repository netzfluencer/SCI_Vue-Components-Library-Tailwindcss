\chapter{Kontext der Arbeit}
\label{chaKontext der Arbeit}
\section{Über My.PORTAL}
Mit My.PORTAL bietet die AVACO GmbH eine anpassbare Webplattform, die als Handels-,
Kommunikations- und Kollaborationslösung in unterschiedlichen Nutzungskontexten eingesetzt werden kann. Besonders hervorgehobene Einsatzzwecke sind die Nutzung als Stadtportal, Unternehmensportal und Bildungsportal, in denen die Plattformbetreiber eine eigene für sie angepasste DSGVO-konforme Portallösung erhalten. \citep{AVACOWebsite}

\section{Architekturen und Technologien}
\subsection{Kommunikation} % (fold)
Das in .NET implementierte Backend basiert auf einer Microservices-Architektur in welcher jeder Microservice gesondert voneinander agiert.  
Zu Projektbeginn wurden Services über RESTful APIs zugänglich gemacht. Seit Ende 2018 werden neue Services mit GraphQL-API eingeführt, weshalb das System derzeit noch in beiden Weisen kommuniziert.

\subsection{Frontend} % (fold)
Das Frontend wird als Vue Single Page Application in einem modularisierten Stil entwickelt. Entscheidend für die Einführung von Vue.js waren laut Angaben der AVACO GmbH die Entwicklungsfreundlichkeit, die solide Community, die Performance, die vielen nützlichen Bibliotheken, die leichte Erlernung sowie die Unternehmensunabhängigkeit des Frameworks gewesen.

Derzeit besteht das Frontend aus den folgenden Modulen:
\begin{itemize}
 \item \textbf{portal-frontend}: Haupt-Applikation, die außerdem auch alle Instanzkonfigurationen beinhaltet.
 \item \textbf{portal-global-components}: Bibliothek von wiederverwendbaren Vue-Komponenten
 \item \textbf{portal-tracker}: Framework um Aktivitäten dem Backend zu übermitteln
 \item \textbf{portal-statistics}: Weitere Vue-Applikation um Portal Administratoren Nutzungsstatistiken anzuzeigen
\end{itemize}
