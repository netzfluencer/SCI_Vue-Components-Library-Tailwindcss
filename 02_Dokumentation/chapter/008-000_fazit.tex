\chapter{Fazit}
\label{cha:Fazit}
Durch die Einführung von Tailwind CSS zusammen mit der neuen Komponentenbibliothek, wurden viele Herausforderungen in der Frontendarbeit von My.PORTAL angegangen.

Komponenten, die auf Tailwind CSS basieren sind nun im Hinblick auf ihre CSS Gestaltung leichter zu überblicken, da die Styledefintionen direkt im Markup nachvollziehbarer sind. Dynamisches setzten von Styleattributen ist durch Utility Klassen leichter geworden und insgesamt müssen kaum noch neue CSS Klassen erstellt werden. Das extrahieren von Markupabschnitten ist (ohne Logik-Teil) nun einfacher möglich und beugt zudem redundanten oder ungenutzten CSS Code vor. Insbesondere Mixins die vorher an unterschiedlichen Stellen vermehrt für ähnliche Zwecke verwendete wurden hatten viel redundanten Code erzeugtet und können nun durch Utility Klassen vollständig ersetzt werden.

In einem stetig wachsenden Projekt wie My.PORTAL war es bisher der Regelfall das mit jeder neuen Funktionalität auch neuer CSS Code erstellt wurde. Durch Tailwind CSS wird auf einer CSS Basis gearbeitet, die kaum weiter wächst und sich sehr gut komprimieren lässt.

Dadurch, dass die Komponentenbibliothek nun unabhängig von einer Portalkonfiguration nutzbar ist und stattdessen auf einer Tailwind-Konfiguration basiert, kann das Frontendteam robuste Vue-Komponenten erstellen, die unabhängig vom Projekt auch in anderen Projekten eingesetzt und angepasst werden können.

Insgesamt konnten in diesem Projekt alle Herausforderungen positiv verändert werden. Jedoch ist die Einführung von Tailwind CSS besonders in einem laufenden Projekt auch mit Risiken, die zusätzlicher Arbeit bedürfen, verbunden. So musste durch die neuen Reset-Styles eine umfangreiche visuelle Korrektur im Frontend vorgenommen werden. Zudem wirkt sich die parallele Verwendung mit zusätzlichen Frontendtechnologien wie SASS auf die Build-Performance des Frontends aus und ist durch die Utility-First-Prinzipien von Tailwind CSS langfristig auch nicht zu vertreten. Ein umfangreiches Refactoring des SASS Codes zu Tailwind CSS hat deshalb bereits begonnen und verfolgt das Ziel den SASS-Preprozessor komplett aus dem Projekt zu entfernen.

Weitere neue Herausforderungen, Chancen und Risiken ergeben sich außerdem auf die bevorstehenden Technologie-Upgrades. Ein Storybook Upgrade auf Version 6 kommt für dieses Projekt erst mit Einführung einer besseren offiziellen automatisierten Vue-Dokumentation in Frage und Tailwind CSS Version 2 wird auf Node Version 12 laufen müssen.

Neue Chancen für besser organisierte Komponenten ergeben sich durch die Einführungen der neuen Composition API in Vue.js Version 3.
