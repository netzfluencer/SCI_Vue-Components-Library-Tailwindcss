\chapter{Einleitung}
\label{cha:Einleitung}

\begin{quotation}
	\emph{``While the utility-class-focused approach of new frameworks like Tailwind CSS make you question everything you know about writing “proper” semantic CSS, its 81\% satisfaction ratio means that it might be time to reconsider our old preconceived notions…''}
	\citep{StateCSS2019_Frameworks}
\end{quotation}

In der aktuellen Studie \textit{The State of CSS 2019} haben Raphaël Benitte und Sacha Greif herausgestellt, dass das neuartige Utility-Frist CSS-Framework \textit{Tailwind CSS} neben allen anderen verglichenen Frameworks die höchste Entwicklerzufriedenheit verzeichnen konnte.

Seit 2017 entwickelt die AVACO GmbH das anpassbare Webportal \textit{My.PORTAL}, das als Handels-, Kommunikations- und Kollaborationslösung in unterschiedlichen Nutzungskontexten eingesetzt werden kann. Dabei dienten bisher das Framework Bootstrap und der CSS-Preprozessor \textit{Syntactically Awesome Style Sheets} (i.F. SASS) als Hauptwerkzeuge für das Styling der wachsenden Vue-Applikation. Mit dem kontinuierlichen Wachstum begegnete das Frontend-Team neuen Herausforderungen in der Frontendentwicklung und insbesondere in der Gestaltung von wiederverwendbaren Vue-Komponenten sowie Style-Elementen. Diese Herausforderungen sind insbesondere in den Bereichen des Komponentenmanagements, der Komponentenentwicklung und des Komponentenstylings zu identifizieren und sollen nun, durch die Einführung einer neuen Komponentenbibliothek sowie Tailwind CSS und PostCSS als neue Gestaltungsbasis, gelöst werden.

Diese Arbeit befasst sich mit der Erstellung der neuen Vue-Komponenten-Bibliothek und der Einführung von Tailwind CSS im Frontend der aktuellen Vue-Applikation.
Dabei sollen Komponenten aus der Bibliothek isoliert bzw. unabhängig vom Hauptprojekt entwickelt, gestaltet und wiederverwendet werden können. Langfristig soll die neue Komponentenbibliothek fähig sein die bereits existierende Komponentenbibliothek zu ersetzten.

Dazu sollen zunächst die Herausforderungen und Probleme in der aktuellen Frontendarbeit von \textit{My.PORTAL} herausgestellt werden. Anschließend erfolgt eine Ermittlung der Anforderungen an die neue Komponentenbibliothek auf dessen Grundlage ein passendes Framework ausgewählt werden soll. Desweiteren sollen die Chancen und Risiken die mit der Einführung von Tailwind CSS verbunden sind ermittelt werden um dann schließlich Tailwind CSS im gesamten Frontend einzuführen.

Anschließend erfolgt eine Bewertung der Implementierung anhand aller vorher herausgestellten Herausforderungen, Anforderungen, Chancen und Risiken. Neue Herausforderungen, Chancen und Risiken werden dargelegt um schließlich auch einen Ausblick auf die neue Vue Version 3 in Bezug auf diese Arbeit anfertigen zu können.