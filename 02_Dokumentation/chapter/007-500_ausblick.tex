\chapter{Ausblick auf Vue.js Version 3}
\label{cha:vue3}

Mit einem Upgrade auf Vue Version 3 kommt es zu einigen Veränderungen und neuen Features für Vue-Entwickler. 

Ein Feature welches optional aber für den Kontext dieser Arbeit besonders relevant ist, ist das Composition API. Dieses wurde für die Entwicklung von Komponenten in größeren Vue-Applikationen gestaltet und stellt eine Alternative zu der weiterhin verfügbaren Options API dar.
Das Composition API wurde insbesondere im Hinblick auf die zwei folgenden Probleme entwickelt \citep{Vue3DocsLogicReuse}: 

\begin{itemize}
  \item \textbf{Komplexe Komponenten verstehen}: Komponenten mit viel JavaScript-Code werden in der Options API nach Optionen kategorisiert. Für komplexe Komponenten ist es jedoch oft verständlicher die Komponente nach logischen Abschnitten zu organisieren.
  \item \textbf{Keine einfachen Mechanismen für Code-Extrationen}: Es gibt keine optimalen Möglichkeiten Code aus Komponenten so zu extrahieren um ihn in anderen Komponenten zu verwenden.
\end{itemize}

Beide Probleme sind auch die Hauptursachen warum in die Post-Modularisierung von Komponenten (vgl. Abschnitt \ref{subsec:postSplitting}) erschwert wird. Mit der Einführung der \textit{setup()} Methode und logischen Strukturierung des Codes würde somit ein Mechanismus entstehen, der das extrahieren von Code erleichtert und generell zu einer anderen Struktur und besseren Lesbarkeit des Komponenten-Codes führt.
