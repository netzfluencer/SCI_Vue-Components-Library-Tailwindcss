\chapter{Arbeitsergebnis}
\label{cha:Arbeitsergebnis}

Für einen Abschluss des Projektes soll das Framework Tailwindcss in allen Fontend-Repositories nutzbar sein und ein neues im Production Stage nutzbares Vue-Komponenten-Repository angefertigt werden. Dieses Repository soll eine Verbesserung gegenüber des bereits existierenden \textit{portal-global-components} - Repository darstellen um dieses langfristig vollständig ablösen zu können. Der Verbesserungsgrad soll anhand der in \autoref{Projektrelevante Herausforderungen im Frontend} genannten Herausforderungen bewertet werden. Außerdem sollen Chancen und Risiken bzgl. der Nutzung von Tailwindcss in einem Vue (v.2) Projekt dargelegt werden und ein Ausblick auf die Nutzung in Bezug auf die neue Vue-Version 3 gegeben werden.